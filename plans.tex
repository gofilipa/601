% Created 2025-03-11 Tue 18:00
% Intended LaTeX compiler: pdflatex
\documentclass[11pt]{article}
\usepackage[utf8]{inputenc}
\usepackage[T1]{fontenc}
\usepackage{graphicx}
\usepackage{longtable}
\usepackage{wrapfig}
\usepackage{rotating}
\usepackage[normalem]{ulem}
\usepackage{amsmath}
\usepackage{amssymb}
\usepackage{capt-of}
\usepackage{hyperref}
\author{fcalado}
\date{\today}
\title{}
\hypersetup{
 pdfauthor={fcalado},
 pdftitle={},
 pdfkeywords={},
 pdfsubject={},
 pdfcreator={Emacs 29.3 (Org mode 9.6.15)}, 
 pdflang={English}}
\begin{document}

\tableofcontents

\section{week 7, archives}
\label{sec:org005d554}

Goals:
\begin{itemize}
\item adjusting some aspects of our class discussion, to bring in
discussion posts, sharing with class
\item second half of class, will start with brainstorming topics for the
final paper project. Start with freewriting on our favorite
readings, and why.
\end{itemize}

Introductions (10m)

Discussion post Pair \& Share (20-25)
\begin{itemize}
\item groups of 3-4: what did you write about? (10)
\item choose one, turn into discussion question (5)
\item share with class (5-10)
\item write on the board
\end{itemize}

Discuss (30)

BREAK

\begin{itemize}
\item introduce critical analysis assignment (10 m)

\item brainstorming final projects 50 m - 1 hr: 
\begin{itemize}
\item freewrite: 5 minutes
\begin{itemize}
\item What was the most interesting reading of the course so far? 
\begin{itemize}
\item what about these did you find interesting?
\end{itemize}
\end{itemize}
\item share with the class, keywords (45 m)
\begin{itemize}
\item going around the room and sharing what you liked and why - 2
min maximum
\item the class helps to come up with 2 keywords for that reading
\end{itemize}
\end{itemize}
\end{itemize}

(THERE WAS NO TIME)
Digital Collections Interface individual activity (10m):
\begin{itemize}
\item choose an interface: LHA, NYPL, NYC Municipal Archives
\item try to find the digital archive section, and browse through the
items. 
\begin{itemize}
\item what can you guess about the way that documents are organized?
\item what does the archive prioritize?
\end{itemize}
\item Share (5-10m)
\end{itemize}

(THERE WAS NO TIME)
In groups, organized by archive (10m):
\begin{itemize}
\item what ways can the online interface be designed that speaks to some
of the aspects from our discussion?
\begin{itemize}
\item community-driven or oriented collections
\begin{itemize}
\item folksonomies
\end{itemize}
\item serendipitious discoveries (elevating more marginalized objects)
\end{itemize}
\item Discuss (15m)
\end{itemize}

\subsection{readings}
\label{sec:orgfdd58bb}
\subsubsection{McKinney, Cait. 2015. Body, Sex, Interface: Reckoning with Images at the Lesbian Herstory Archives. Radical History Review 122: 115–28.}
\label{sec:orgbd83959}
About the work being done at a community, volunteer run archives.

Asks a central question, very related to Hartman, which is: \textbf{how do we
catalog things, make them visible and accessible, without also
incorporating/subscribing them into a normalizing scheme}? How do we
label things and allow them to maintain a certain kind of dynamicity?

Much of the data in the archives cannot be fully
categorized/catagloged and digitized, because (1) we never got that
information from the donor, we take everything and anything,
"undescribability" and lack of donor agreements, and (2) we don't have
the resources to do all of that work.
\begin{itemize}
\item folksonomies
\item "the ways that all kinds of sex practices and gendered ways of being
scramble the categorical logics of structured databases" (3).
\end{itemize}

Designing for serendipitious experiences (the interface): 
\begin{itemize}
\item What kind of interface would replicate some of the "uncategorizable"
aspects of the materials?
\item "Pulling a “what do you say about this?” image out of the photo
drawer evokes wonder, because the ways these photos do not make
sense are difficult to catalog and capture through mechanisms such
as the searchable database form" (10).
\end{itemize}

Being visible vs being integrated/maintreamed:
\begin{itemize}
\item There is a desire for access, which is good. But being seen also
suggests being included, and being included into what kind of
citizen:
\item "LGBT archives are worlding technologies that can be called on to
support homonational trends, in which the recognition of gay and
lesbian citizen-subjects as rightly historical is tied to broader
political agendas of gendered and racialized violence, exclusion,
and empire in the present. Photographic archives, in particular,
shift this politics into a regime of visibility that associates
being seen with being welcomed into the fold of liberalism" (8).
\end{itemize}

\subsubsection{Hartman, Saidiya. "Venus in Two Acts." Small Axe, vol. 12 no. 2,   2008, p. 1-14. Project MUSE muse.jhu.edu/article/241115.}
\label{sec:org0cb7c01}

Her main question is how can we write history under these conditions
of scarcity/absence and of language: 
\begin{itemize}
\item “How does one revisit the scene of subjection without replicating
the grammar of violence?” (4).
\end{itemize}

"The violence of the archive" - we only receive things in the violent
terms of their subjection. 
\begin{itemize}
\item "The archive of slavery rests upon a founding violence. This
violence determines, regulates and organizes the kinds of statements
that can be made about slavery and as well it creates subjects and
objects of power" (10).
\begin{itemize}
\item A condition also known as the "violence of the archive," she
describes the archive as a "death sentence," because it only
records the subject in the terms of their objectification, in "a
display of the violated body, an inventory of property" (2).
\end{itemize}
\item Hartman here looks at the problem of what to do with an absent
archive. Not only absence in the form of evidence, that the literal
records are missing, but also in the tools of expression, in
language that cannot approximate the reality of experience, and in
the discourse that dictates silence.
\end{itemize}

She seeks to recuperate (without recovering) the lives of these
subjects. To write about them in a way that does not do more damage,
but draws attention to the ways that their lives have been delineated
while inviting possibility for living. To create in the mode of
"critical fabulation" (11).

She examines the history of Venus, the unnamed slave woman who appears
variously throughout the "official" record. From this history, Hartman
concludes that there is no way forward with recovery. She turns to
consider a series of paradoxical questions:
\begin{itemize}
\item “how does one rewrite the chronicle of a death foretold and
anticipated, as a collective biography of dead subjects, as a
counter-history of the human, as the practice of freedom?” (3).
\item "how does one recuperate lives entangled with and impossible to
differentiate from the terrible utterances that condemned them to
death, the account books that identified them as units of value, the
invoices that claimed them as property, and the banal chronicles
that stripped them of human features?" (3)
\item “How can narrative embody life in words and at the same time respect
what we cannot Know?” (3).
\item “If it is no longer sufficient to expose the scandal, then how might
it be possible to generate a different set of descriptions from this
archive?" (7).
\end{itemize}

The archivist of slavery comes up against the incommensurability
between reality and the historical record, the archivist must endeavor
to engage this incommensurability: "to expose and exploit the
incommensurability between the experience of the enslaved and the
fictions of history, by which I mean the requirements of narrative,
the stuff of subjects and plots and ends" (10).
\begin{itemize}
\item "This double gesture can be described as straining against the
limits of the archive to write a cultural history of the captive,
and, at the same time, enacting the impossibility of representing
the lives of the captives precisely through the process of
narration" (11).
\end{itemize}

In the scarcity of material (not one autobiographical account of a
female survives), most of what we have left are numbers. Can we then
fill the void with stories? 
\begin{itemize}
\item "Loss gives rise to longing, and in these circumstances, it would
not be far-fetched to consider stories as a form of compensation or
even as reparations, perhaps the only kind we will ever receive"
(4).
\end{itemize}

\subsubsection{Drabinski, Emily. “Queering the Catalog: Queer Theory and the Politics of Correction.” The Library Quarterly: Information, Community, Policy, vol. 83, no. 2, 2013, pp. 94–111}
\label{sec:org8eb85e8}

What does queer theory give to cataloguing? It gives a perspective
that nothing will ever be perfectly described. That engaging with
materials is a dialogical process, that markup will always be complex
and biased.

"dialogic engagement" - a back and forth engagement with sources,
rather than a one-way engagement.

Narrative around Gay and Lesbian Studies vs Queer Theory.
\begin{itemize}
\item G\&L was about filling a gap, giving voice to an absence, and QT was
about outlining the queer subject.
\end{itemize}

The location of where queer and trans books have been cataloged, as
sexual deviance, trans next to homosexuality, suggesting certain kinds
of relationships between them.

The subject headings themselves containing bias.

Queer theory assumes that all categorizing schemas are contingent to
history and context. Nothing is universal.

Queer is always in resistance. It needs a norm in order to define
itself against. "Lesbian should be replaced by Dyke" (103).

Solutions:
\begin{itemize}
\item "Turning library access structures into pedagogical tools"
\item making tagging system / links visible
\item user tagging
\end{itemize}

\subsubsection{“How to become a pirate archivist,” by Anna’s Archive. 10/22/2017. \url{https://annas-archive.org/blog/blog-how-to-become-a-pirate-archivist.html}}
\label{sec:orgc44d5e3}
\begin{itemize}
\item they are pirates, not bound by law, but only by an imperative to
make all information accessible by scraping collections on the
internet and sharing them.
\item they are lonely.
\item what are their strategies for the archival work?
\begin{itemize}
\item data scraping
\item metadata gathering
\item mirroring collections
\item hosting the content
\item seeding the content
\end{itemize}
\end{itemize}
\end{document}
